\documentclass[11pt,a4paper]{article}

\usepackage{amsmath,amssymb,amsfonts,verbatim}
\usepackage[margin=1.5cm, vmargin=2cm]{geometry}
\usepackage{graphicx}
\usepackage{xcolor}
\usepackage{commath}
\usepackage{hyperref}
\usepackage{float}
\usepackage{array}

\def\N{\mathbb{N}}
\def \Z {\mathbb{Z}}
\def \Q {\mathbb{Q}}
\def \R {\mathbb{R}}

\usepackage{fancyhdr} 
\pagestyle{fancy}
\fancyhead{}         
\fancyhead[C]{Assessing and communicating the mitigation of river floods} 
\fancyhead[L]{MATH 3001}     
\fancyhead[R]{2018/19} 

\begin{document}

\setlength{\parindent}{0cm}

\begin{titlepage}
\begin{center}
Mary Saunders\\
Student ID: 201001740\\
\vspace{2cm}
{\huge \textbf{MATH3001: Project in Mathematics}}\\
\hrulefill

\vspace{1cm}
{\LARGE Flood analysis: Assessing and communicating the mitigation of river floods to policy makers and the general public}\\
\vfill
\end{center}
\end{titlepage}

\tableofcontents 
\noindent \hrulefill

\newpage

\section{Introduction}
"Major flooding in the UK is now likely to happen every year but ministers still have no coherent long-term plan to deal with it" \cite{1}. A major flooding event is not something that everyone tends to worry about in the UK, however they appear to be occurring more and more frequently, with disastrous consequences. Extreme floods leave families without homes, businesses and sometimes more devastatingly, without family members. The floods of Boxing Day in 2015 alone caused damage costing billions of pounds across the North of England, with the floods inducing two explosions, multiple fires and the partial destruction of a 200-year old listed building in Greater Manchester \cite{2}.  \\

The aim of this project is to analyse flood data for various rivers provided upon request by the Environment Agency, to look at existing flood mitigation plans and use these to propose some of our own, whilst also looking at the cost-effectiveness of these. Overall we aim to assist policy makers and the general public in ways that may help to reduce the damaging effects of extreme river events.\\

A brief outline of this report is as follows:
\begin{itemize}
\item Different ways to think about the mitigation of floods
\item Analysis of the 2015 Boxing Day floods in Yorkshire and Greater Manchester, and other floods across the North of England
\item Mitigation schemes for these areas
\item Cost-effectiveness of these schemes 
\end{itemize}

\section{Flood mitigation}

\subsection{What is meant by flood mitigation?}
Flooding is brought about by the combination of heavy rainfall and lack of flood defences. No matter how hard anyone tries to stop certain areas flooding, no one can change the weather. Therefore mitigation schemes are implemented in order to try to reduce the severity of the damage caused by flooding in a given area. Mitigation is planning actions and the installation of physical barriers in order to reduce the amount of flooding and its consequences in the future. This can be done through many different measures, from natural and structural solutions to simply increasing local knowledge and awareness of how floods behave in the area. 

\subsection{Examples of mitigation schemes}
There are many approaches to flood mitigation in use, some more effective than others. Structural mitigation measures are put into place to counteract the flood event, and include: 
\begin{itemize}
\item Dams - dams impound floodwater and either divert the water to be used elsewhere, or control the release of it into the river below.
\item Raising river banks/high flood walls, and river bed widening - these methods increase the capacity of the river and therefore the amount of water it can hold before flooding occurs.
\end{itemize}
There are also multiple natural flood mitigation (NFM) strategies available, which alter, restore or make use of the surrounding landscape to reduce flood risk \cite{3}. These ideas include:
\begin{itemize}
\item Tree planting - trees absorb some rainfall, albeit a minute amount, but also directly intercept rainfall. Trees also encourage high absorption rates in the soil.
\item Peatland restoration - this trenching technique provides a barrier that stops water from draining away through underground cracks, and so reduces river peak flows \cite{4}.
\end{itemize}
Homeowners themselves can also implement certain mitigation schemes to reduce risk of damage from flooding. These include:
\begin{itemize}
\item Flood insurance - the purchase of home and contents insurance specific to damage caused by flooding ensures that homeowners are not as affected by the loss of any belongings.
\item Relocation - a simple solution is to relocate out of the area at risk of flooding.
\end{itemize}

\subsection{Flood excess-volume (FEV)}
In order to effectively plan and analyse flood mitigation schemes, we need to know how much flood water there is. To think about this, we can use the term \textit{flood-excess volume}. Flood-excess volume (FEV) is simply the volume of water that causes the flood in question, and so is the amount we wish to mitigate/reduce to zero using however many mitigation measures, to prevent flood damage for an event within a given return period. Note that here, a return period is the probability that a flood will occur within any given year. For example, if a flood has a return period of 100 years (probability of 1/100), this means that in any given year the probability of such a flood occurring is 1\%. This is frequently misinterpreted to mean that if a flood with this return period occurs then the next flood will happen in 100 years \cite{5}.\\

In order to calculate the FEV, we must first obtain data from the Environment Agency for our desired flood. River levels are checked at a regular interval of 15 minutes during the year by a large network of river gauges throughout the country that measure the river height at the gauge location. The data recorded by these monitoring stations is available freely upon request by the Environment Agency. \\

From the data collected, we must first take the threshold height of the river above which flooding occurs ($h_T$). This can be estimated from social media pictures of the local area at the same time of the flood and can be compared to readings obtained from the monitoring station at the same given time. Conversely, online resources such as Shoothill's GaugeMap (http://www.gaugemap.co.uk/) can be used which provides heights of rivers above which flooding is possible in an area may occur. \\

To convert $h_T$ into a volume of water (or discharge), a relationship between the two must be established. This association is known as a rating curve and is created by taking frequent measurements from a specific monitoring station at a specific time and plotted on a graph, ie. $Q = Q(\bar{h})$, where $Q$ is the flow of the water. \\

We can convert our $h_T$ into a value for the threshold discharge, $Q_T=Q(h_T)$. Then the FEV, or $V_e$, is defined as the integral of $Q(t)-Q_T$ for the duration $T_f$ of the flood, which itself is when $Q-Q_T > 0$. $V_e$ is given in units $m^3$. \\

\textcolor{red}{****NEED TO REFERENCE PROJECT DESCRIPTION HERE****}

\subsection{Square lake graphs}
Having found the FEV for a given flood, it can be hard to visualise just how much floodwater needs to be mitigated. A simple yet effective way to express this value is as a 2-metre deep 'square lake', with side lengths relevant to the value of the FEV, so that the volume of the square lake matches that of the flood-excess volume. This immediately makes it much easier to imagine how much flood water needs to be mitigated, and as the depth is minimal compared to the length of the sides, this allows us to view the lake from above as a square. This in turn makes it much simpler to allocate certain proportions of the flood-excess volume floodwater to different flood mitigation strategies \cite{6}.\\

\textcolor{red}{****Include image of standard square lake template****}\\

\section{Boxing Day 2015 floods in Yorkshire}
To start our project, we initially had to recreate the quadrant plots for the Yorkshire floods already previously created by Professor Onno Bokhove and Dr Thomas Kent, our supervisors, and provide our own analysis. 

\subsection{The River Aire, Leeds}
The River Aire begins its course at Malham Tarn, a lake in the Yorkshire Dales, and flows through Leeds before it joins the River Ouse in the village of Airmyn \cite{7}. Intense rainfall over the Christmas period in 2015 caused the Aire to break its banks and cause damage to 3,355 properties, including homes and businesses. Before this event, Leeds' largest flood was in 2000, when only 100 properties were affected - the significant increase in damage massively highlighted Leeds' need for more effective flood defences \cite{8}.\\

For the analysis of this flood in Leeds, we looked at the monitoring station for the River Aire at Armley. The average water depth at Armley in standard weather conditions is between 0.28m and 0.95m, and as we will see in the following analysis, the Aire reached its highest ever recorded level of 5.21m during the Boxing Day flood \cite{9}. This figure alone shows what a catastrophic and damaging flood this was.

\subsubsection{Quadrant plot and analysis}
Having received data from the Environment Agency outlining the height ($m$) and the flow ($m^3/s$) for the River Aire over the period of the flood, we were able to plot a quadrant graph which shows different relationships between the variables (figure 1). River level measurements at regular 15 minute intervals between 25th December and 30th December 2015 were used here. Note that here, flow means the volume of water that passes a specific point per second. I used the program R to plot the following graph, and all other graphs included in this report.
\begin{figure}[H]
\centering
\includegraphics[scale=0.45]{airegraph.png}
\caption{Quadrant plot for the 2015 Boxing Day flood of the River Aire at Armley}
\end{figure}
The lower left quadrant shows river height against time, the upper right quadrant shows time against flow, and the upper left quadrant contains a rating curve (explained in further detail in section 3.1.3). We only used the raw data for height and time for this graph - flow was not used for the end result, as the rating curve provided us with more accurate data.

\subsubsection{Data scaling and plotting of graph}
When first plotting the graph, my first instinct was to create four separate subplots and attempt to merge them together, however I soon realised that this was not feasible. Instead, as a group we decided the best way to move forward with our recreations of the graph would be to treat the data as one data set. We scaled the height ($h$) and time ($t$) data so that all the data points were between 0 and 1, allowing us to create one singular plot and add each curve to it separately. \\

To plot the rating curve in the upper left quadrant, we originally plotted the raw data for height against the raw data for flow. This provided us with a similar curve, however we soon learnt that not all monitoring stations in the UK actually measure the flow of the river, and that the Environment Agency frequently get their flow data from a rating curve formula.\\

In the lower left quadrant we simply plotted the raw data for height against time. The rating curve was plotted using the function explained in the following section, and the curve in the upper right quadrant shows the relationship between flow and time - this was plotted by entering the raw $h$ values into the rating curve function and plotting these against $t$. It is important to note here that the discharge curve, $Q$, is a composite of functions - $Q$ is a function of $h$, which is in turn a function of $t$. That is,
$$Q(t)=Q(\bar{h})=Q(\bar{h}(t)).$$

\subsubsection{The rating curve}
In hydrology, a rating curve establishes a relationship between river height and the discharge of water. It is developed by taking measurements for the discharge at monitoring stations, which can fluctuate over time, so it is necessary to reasses at various points in time \cite{10}. The rating curve is plotted using the function: 
$$Q(h) = c(h-a)^b$$
where $h$ takes arbitrary values of regular intervals between 0 and 6, and $Q$ is flow. We chose 6 as the upper limit as it is higher than the maximum height reached by the river, so allows for extrapolation. The coefficients $a,b$ and $c$ are provided for each monitoring station in a rating change report, available on request from the Environment Agency. For Armley, the rating change information is as follows: \\
\begin{table}[H]
\centering
\begin{tabular}{| c c c c c |}
\hline
\textbf{Lower Stage Limit {(}m{)}} & \textbf{Upper Stage Limit {(}m{)}} & \textbf{c} & \textbf{b} & \textbf{a} \\
\hline
0.2 & 0.685 & 30.69 & 1.115 & 0.156 \\
0.685 & 1.917 & 27.884 & 1.462 & 0.028 \\
1.917 & 4.17 & 30.127 & 1.502 & 0.153 \\
\hline
\end{tabular}
\caption{Table showing the coefficients for the rating curve function for different values of $h$ for the River Aire at Armley}
\end{table}
\vspace{\baselineskip}
This information allowed me to plot the rating curve, and from that a linear approximation of the curve (shown in figure 1 by the dashed line in the upper left quadrant).

\subsubsection{Estimation of $h_T$ and calculating $h_M$}
In this case, $h_T$ was estimated as $3.9$m by Professor Bokhove, by accessing the timestamp on a photo of the Aire just as it started to flood, and comparing this with the data obtained form the Environment Agency, to see the measured river height at this time. \\

From this, $h_M$ (the average of all river heights above $h_T$) was calculated by taking all data points above $h_T$, and finding the mean of these, to find the mean height of the flood: \\
$$h_M = \frac{\sum_{i=1}^{k} h_k}{k} \text{ , for } h_k \geq h_T$$

\subsubsection{Calculating FEV}
From $h_T$ and $h_M$, and using the rating curve, we can relate a threshold discharge $Q_T$ to $h_T$, that is $Q_T=Q(h_T)$. Similarly, we can find a relating mean discharge $Q_M$ to $h_M$, that is $Q_M=Q(h_M)$. These values then allow us to calculate the flood-excess volume and the duration of the flood, $T_f$.\\

The first estimation for the FEV ($V_e$) is given by the black rectangle in the upper right quadrant, and is calculated by:
$$V_e = T_f(Q_M - Q_T)$$
where $T_f$ is given by the time difference between when the river level first passes the threshold $h_T$, and when it then drops back below it.\\

Another estimate is shown by the shaded region which is labelled 'FEV' in figure 1, and is calculated by:
$$V_e=\sum_{k=1}^{N_m} (Q(\bar{h}_k)-Q(h_T)) \Delta t.$$

Therefore using these calculations, the estimation for the FEV for the flood of the River Aire is $9.34$M$m^3$, with $T_f$ being 32 hours.

\subsubsection{Flood mitigation schemes in Leeds}
Since the December 2015 floods in Leeds, the Leeds Flood Alleviation Scheme has been put into place by the Leeds City Council in partnership with the Environment Agency, and has been split into two phases. Phase 1 began in January 2015 and was completed in October 2017, costing \pounds50 million, and included movable weirs, merging canals and rivers, and long stretches of floods walls \cite{11}.\\

Phase 2 takes into account the entire catchment area of the River Aire, and combines natural flood management strategies with structural mitigation measures further upstream, to reduce the risk of flooding downstream. The measures currently under discussion include the creation of new woodland areas, the construction of flood storage areas, and the removal of objects along the River Aire which may currently cause higher than necessary river levels \cite{11}.\\

Overall the scheme will provide greater protection to over 3000 properties to ensure the disastrous consequences of the 2015 floods will not happen again.

\subsubsection{Square lake graph}
One scenario from the Leeds Flood Alleviation Scheme Phase 2 involves building flood walls of a height of 1.6m in the city centre. Constraining water within the river channel would help to prevent flooding, however could cause flooding further downstream, therefore the construction of a flood storage area upstream in the Calverley floodplain could counteract this. The capacity of the floodplain can also be increased by installing movable weirs, to temporarily store floodwater \cite{11}. A square lake graph for this scenario is as follows:
\begin{figure}[H]
\centering
\includegraphics[scale=0.4]{airesquarelake1.png}
\caption{Square lake graph for the Flood Walls {\&} Calverley storage scenario}
\end{figure}

This plot shows that 8{\%} of the flood-excess volume will be mitigated by the Calverley storage, with the remaining 92{\%} by the 1.6m flood walls. It also shows the cost-effectiveness of the mitigation strategy - \pounds0.75M per 1{\%} of FEV mitigated, which includes \pounds10M for the storage area, and \pounds65M for the flood walls.

\subsection{The River Calder, West Yorkshire}
The River Calder originates in Heald Moor and flows through the Pennines before joining the River Aire at Castleford \cite{12}. Over 6000 properties in and around the Calder Valley were flooded during the Boxing Day floods, causing a loss to Calderdale's economy of \pounds47 million. These floods prompted the Council to organise a live training event to prepare local teams for rescues during floods. The Calder Valley has flooded nearly every year of this decade, showing that new flood mitigation strategies would be welcomed \cite{13}.\\

For the analysis of the River Calder flood, the monitoring station in Mytholmroyd was chosen. The average river height at Mytholmroyd in average weather conditions is between 0.43m and 2.80m, and as shown in the following quadrant plot, the highest ever recorded level of 5.65m occurred during in the Boxing Day flood \cite{14}.

\subsubsection{Quadrant plot and analysis}
I plotted the following quadrant graph (figure 2) in the same way as I plotted the graph for the River Aire, by scaling the data set and plotting each curve separately.  I used the data provided by the Environment Agency for the time period of 25th December to 29th December 2018, and again only height and time raw data were used.
\begin{figure}[H]
\centering
\includegraphics[scale=0.45]{caldergraph.png}
\caption{Quadrant plot for the 2015 Boxing Day flood of the River Calder at Mytholmroyd}
\end{figure}

\subsubsection{The rating curve}
The rating curve and its linear approximation for the River Calder were plotted in the same way as that for the River Aire, using the following coefficients for the function provided by the Environment Agency:\\

\begin{table}[H]
\centering
\begin{tabular}{| c c c c c |}
\hline
\textbf{Lower Stage Limit {(}m{)}} & \textbf{Upper Stage Limit {(}m{)}} & \textbf{c} & \textbf{b} & \textbf{a} \\
\hline
0 & 2.107 & 8.459 & 2.239 & 0.342 \\
2.107 & 3.088 & 21.5 & 1.37 & 0.826 \\
3.088 & 5.8 & 2.086 & 2.515 & -0.856 \\
\hline
\end{tabular}
\caption{Table showing the coefficients for the rating curve function for different values of $h$ for the River Calder at Mytholmroyd}
\end{table}

\subsubsection{Calculating FEV}
The FEV for the Boxing Day flood of the River Calder was calculated and is shown on the quadrant plot in the same way as the River Aire, and was estimated as 1.65M$m^3$, with $T_f$ as 8.25 hours. This FEV is much smaller than that of the River Aire, but this comes as no surprise as the flood duration for the Calder is much shorter than the Aire.\\

The shaded region in the quadrant plot which depicts the FEV is not quite accurate, as it overspills out of the curve and the FEV rectangle. This is due to an issue I have had with the 'polygon' function in R - I am still working on trying to rectify this, as the $Q_M$ and $Q_T$ values do not actually exist as points in my data.

\subsubsection{Future/ongoing mitigation schemes in Calderdale}
Follwoing the devastating floods of 2015, the Calderdale Flood Action Plan was put into place in January 2016 by the Calderdale Flood Partnership along with the Environment Agency. Part of this scheme involves the strengthening of defences, and looks at using reservoirs and canals to store floodwater and therefore reduce the risk of flooding. The sewer networks in the risk areas are also under observation by Yorkshire Water, as flooding has occurred here in the past and could potentially be prevented in the future \cite{15}.\\

Natural flood management measures have also been considered in the action plan. Grants are available for farmers in Calderdale to carry out plans for NFM schemes, including tree planting, the temporary storage of water and considering the management of farmland to ensure more water is absorbed \cite{15}.\\

The final part of the scheme involves the awareness and resilience  of flooding of the local community. This includes signing up for flood warnings from the Environment Agency, raising awareness in the community of the dangers of flooding and having a team of volunteers available to help out during and after a flood \cite{15}. These measures together with the previously mentioned ones all work together to form a Flood Action Plan that can help mitigate the risk of damage caused by floods in Calderdale.

\subsection{Comparison of the Aire and the Calder}
I will compare the floods of the River Aire and the River Calder, looking at the different FEVs and mitigation schemes. 

\section{The River Don, Sheffield}
The River Don begins in the Peak District and flows through South Yorkshire, before joining with the River Ouse at Goole. In June 2007 the country was hit with a period of vigorous rainfall which caused severe flooding, resulting in 13 deaths and damage to over 50,000 properties. Sheffield was one of the worst-affected cities, with 2 deaths occurring here, hundreds of people being evacuated and many having to be airlifted to safety from the rooftops of buildings \cite{16}.

\subsection{June 2007 flood}
For the analysis of the 2007 flood in Sheffield, we looked at data from the monitoring station for the River Don at Hadfields. In standard weather conditions the river height here is between 0.32m and 0.53m, with the highest ever recorded level being 4.68. This height was reached during this flood, showing just how disastrous it was and how flood mitigation strategies are needed to avoid an event of this magnitude happening again \cite{17}.

\subsubsection{Quadrant plot and analysis}
The following quadrant plot (figure 3) was plotted in R using the same methods as the previous quadrant graphs. The data was provided by the Environment Agency, and I used the raw data for height and time for the period of 25th June to 29th June 2007.
\begin{figure}[H]
\centering
\includegraphics[scale=0.45]{dongraph.png}
\caption{Quadrant plot for the June 2007 flood of the River Don at Hadfields}
\end{figure}

\subsubsection{The rating curve}
The upper left quadrant contains the rating curve, which was plotted using the rating curve function and the following coefficients provided by the Environment Agency.
\begin{table}[H]
\centering
\begin{tabular}{| c c c c c |}
\hline
\textbf{Lower Stage Limit {(}m{)}} & \textbf{Upper Stage Limit {(}m{)}} & \textbf{c} & \textbf{b} & \textbf{a} \\
\hline
0 & 0.52 & 78.4407 & 1.7742 & 0.223 \\
0.52 & 0.931 & 77.2829 & 1.3803 & 0.3077 \\
0.931 & 1.436 & 79.5956 & 1.2967 & -0.34 \\
1.436 & 3.58 & 41.3367 & 1.1066 & -0.5767 \\
\hline
\end{tabular}
\caption{Table showing the coefficients for the rating curve function for different values of $h$ for the River Don at Hadfields}
\end{table}

\subsubsection{Calculating FEV}
The flood-excess volume for the threshold $h_T=2.9$m was approximated at 3.00M$m^3$, with the corresponding value for $T_f$ being 13.5 hours. This threshold $h_T$ was estimated by a value given by the Environment Agency, then slightly raised to ensure that flooding will occur at this level.

\subsection{Future/ongoing mitigation schemes in Sheffield}
I am aware of a large scale flood alleviation plan that is ongoing in the area of Sheffield and Doncaster, which involves flood warning sirens and live training exercises for the rescue of people trapped by floods. I will research this in more detail and comment on the cost-effectiveness of my findings.

\subsection{Comparison of the 3 rivers}
I plan to compare the floods of the River Aire, the River Calder and the River Don, by looking at their FEVs, maximum river heights reached and current status of any ongoing flood mitigation plans, together with their cost-effectiveness.

\section{The River Irwell, Greater Manchester}
Having recreated and conducted our own analysis of Professor Bokhove and Dr Kent's quadrant graphs for the floods of the River Aire, the River Calder and the River Don, we chose a different flood and applied the skills we have acquired to it. I decided to look at the River Irwell in Greater Manchester as it is the nearest large river to where I live. It also flooded on Boxing Day 2015, so I thought it would be interesting to compare this flood with those in Yorkshire.\\

The River Irwell originates at a spring on Deerplay Moor, flows through Greater Manchester forming a boundary between Manchester and Salford, before it joins the River Mersey at Irlam. As it passes many large attractions in the city, such as the cathedral and the Manchester Arena, any flooding that occurs along the river would be extremely disruptive to the local area.

\subsection{Boxing Day 2015 flood}
The heavy rainfall around Christmas 2015 caused the River Irwell to break its banks, causing major flooding to Greater Manchester that left behind devastation. Over 2,200 properites were flooded and 30,000 were left without power. Damage to the local infrastructure cost \pounds11.5 million. \\

For the analysis of this flood in 2015, I chose the monitoring station at Manchester Racecourse for the River Irwell as being central to Manchester, it would be interesting to see how the flooding affected the busiest areas of the city. In normal weather conditions the height of the river at this station is between 0.75m and 1.30m. The highest ever recorded height is 5.67m, which was reached during this flood \cite{18}. The difference between the average height and the highest reached shows the extent of the flood and just how disastrous it was. 

\subsubsection{Quadrant plot and analysis}
In order to plot a quadrant graph showing the relationships between height, time and flow for this flood, I requested the raw data for these variables from the Environment Agency. I received the data for the whole year so had to extract the needed data for the durtaion of the flood, and convert it into a .csv file to be able to import it into R. Once i had done this, I was able to plot the graph using the same methods as the previous graphs.\\

I estimated $h_T$ by looking at Shoothill's GaugeMap. This told me that minor flooding was possible in the area at 3m, and by adding a 25{\%} uncertainty I estimated $h_T$ to be 3.75m, to ensure that flooding will occur in this area. From here, I was able to estimate the mean height of flooding above the threshold as $h_M = 4.93$m. The quadrant plot is as follows:
\begin{figure}[H]
\centering
\includegraphics[scale=0.45]{irwellgraph.png}
\caption{Quadrant plot for the June 2007 flood of the River Irwell at Manchester Racecourse}
\end{figure}

\subsubsection{The rating curve}
The rating curve in the upper left quadrant was plotted directly from the raw data, by plotting height against flow, as the Environment Agency have not yet provided me with the rating change report, which contains the coefficients needed for the rating curve function.

\subsubsection{Calculating FEV}
I have not yet calculated the FEV as I have not been able to plot the rating curve function. I am aware that it is possible to calculate the FEV from knowing only $h_T$ and $h_M$, however I am struggling in R to find the intersection of these values with the rating curve in order to find $Q_T$ and $Q_M$, and therefore find the flood-excess volume. I will continue to work on this as I await the extra information required.

\subsection{Flood mitigation schemes in Greater Manchester}
In February 2018, the construction of a \pounds10 million flood basin was completed at Castle Irwell in Salford. It can hold more than 625,000m$^3$ of floodwater and reduces the risk of flooding from the River Irwell to 2000 properties. The basin will hold river water when a flood warning is issued, and will release it through pipes back into the river when the river height starts to fall again \cite{19}. This flood basin works alongside the already operational flood basin at Littleton Road which can hold up to 650,000m$^3$ of floodwater. The construction of the new basin forms part of a flood management plan that was allocated \pounds22 million in order to reduce the risk to 5000 properties. The plan is aimed to be completed by 2021.

\subsection{Comparison to Yorkshire floods}
I will compare the flood of the River Irwell to those that happened at the same time in Yorkshire.

\section{Future plans for the project}
For the remaining part of the project, I aim to examine at least one more monitoring station along the River Irwell, which will provide me with more information and will enable me to compare the flood-excess volume at different stages in the river. I will then select a further river which has been researched already by another member of my group and attempt to duplicate their findings in order to confirm them.\\

I aim to research more deeply and analyse the available flood mitigation strategies for the rivers I have studied, and look at the cost-effectiveness of each.\\

I could also consider longer time records for the data of the floods and explore the return periods for each, and what this might mean. I could also investigate the basics of the fluid dynamics involved in certain mitigation measures. In addition, I will begin to examine rainfall scenario analyses and investigate spatial rainfall distributions.

\begin{thebibliography}{}
\bibitem{1}The Guardian, 2016: Major flooding in UK now likely every year, warns lead climate adviser. Available: https://www.theguardian.com/environment/2016/dec/26/major-flooding-in-uk-now-likely-every-year-warns-lead-climate-adviser-storm-desmond. Last accessed: 8th December 2018.
\bibitem{2}Wikipedia: 2015 Great Britain and Ireland floods. Available: https://en.wikipedia.org/wiki/2015-16{\_}Great{\_}Britain{\_}and{\_}Ireland{\_}floods. Last accessed: 8th December 2018.
\bibitem{3}Institute of Chartered Foresters, 2017: Trees can Reduce Floods. Available: https://www.charteredforesters.org/2017/06/trees-can-reduce-floods/. Last accessed 6th December 2018.
\bibitem{4}Environment Analyst, 2017: New peatland restoration technique could reduce flooding. Available: https://environment-analyst.com/dis/53991/new-peatland-restoration-technique-could-reduce-flooding. Last accessed: 8th December 2018.
\bibitem{5}NIWA: What is a return period? Available: https://www.niwa.co.nz/natural-hazards/faq/what-is-a-return-period. Last accessed: 7th December 2018.
\bibitem{6}Tom Kent, 2018: Using 'flood-excess volume' to quantify and communicate flood mitigation schemes. Available: https://research.reading.ac.uk/dare/2018/09/27/using-flood-excess-volume-to-quantify-and-communicate-flood-mitigation-schemes/. Last accessed: 10th December 2018.
\bibitem{7}Wikipedia: River Aire. Available: https://en.wikipedia.org/wiki/River{\_}Aire. Last accessed: 10th December 2018.
\bibitem{8}Yorkshire Evening Post, 2017: 2015 Leeds floods: The day that will leave lasting memory. Available: https://www.yorkshireeveningpost.co.uk/news/2015-leeds-floods-the-day-that-will-leave-lasting-memory-1-8926175. Last accessed: 10th December 2015.
\bibitem{9}River Levels: River Aire at Armley. Available: https://riverlevels.uk/river-aire-leeds-armley{\#}.XA512hP7TfY. Last accessed: 10th December 2018.
\bibitem{10}USUS: What is a rating curve? Why does it change over time? Available: https://www.usgs.gov/faqs/what-a-rating-curve-why-does-it-change-over-time?qt-news{\_}science{\_}products=0{\#}qt-news{\_}science{\_}products. Last accessed: 10th December 2018.
\bibitem{11}Leeds City Council: Leeds Flood Alleviation Scheme. Available: https://www.leeds.gov.uk/parking-roads-and-travel/flood-alleviation-scheme. Last accessed: 11th December 2018.
\bibitem{12}Wikipedia: River Calder, West Yorkshire. Available: https://en.wikipedia.org/wiki/River{\_}Calder,{\_}West{\_}Yorkshire. Last accessed: 10th December 2018.
\bibitem{13}BBC News, 2016: Calder Valley flood sirens sound for 'biggest ever' exercise. Available: https://www.bbc.co.uk/news/uk-england-leeds-37654405. Last accessed: 10th December 2018.
\bibitem{14}River Levels: River Calder at Mytholmroyd. Available: https://riverlevels.uk/river-calder-hebden-royd-mytholmroyd{\#}.XA6abxP7TfY. Last accessed: 10th December 2018.
\bibitem{15}Eye on Calderdale, 2018: Reducing the risk of flooding in Calderdale. Available: https://eyeoncalderdale.com/Media/Default/Flooding{\%}20Documents/FAP/Updated-Calderdale-FAP-bklt-2018.pdf. Last accessed: 12th December 2018.
\bibitem{16}The Yorkshire Post, 2017: The fight to protect Sheffield from repeat of deadly floods of 2007. Available: https://www.yorkshirepost.co.uk/news/the-fight-to-protect-sheffield-from-repeat-of-deadly-floods-of-2007-1-8611534. Last accessed: 11th December 2018.
\bibitem{17}River Levels: River Don at Hadfields. Available: https://riverlevels.uk/river-don-tinsley-hadfields{\#}.XBAifxP7TfY. Last accessed: 11th December 2018.
\bibitem{18}River Levels: River Irwell at Manchester Racecourse. Available: https://riverlevels.uk/irwell-broughton-manchester-racecourse{\#}.XBECwhP7TfY. Last accessed: 12th December 2018.
\bibitem{19}Gov.uk, 2018: Environment Agency completes �10 million flood storage basin on World Wetlands Day. Available: https://www.gov.uk/government/news/environment-agency-completes-10-million-flood-storage-basin-on-world-wetlands-day. Last accessed: 12th December 2018.
\end{thebibliography}

\section{Appendix}

\end{document}